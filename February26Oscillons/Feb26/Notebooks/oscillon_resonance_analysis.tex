\documentclass[11pt,a4paper]{article}

% ── packages ──────────────────────────────────────────────────────────────────
\usepackage[utf8]{inputenc}
\usepackage[T1]{fontenc}
\usepackage{amsmath,amssymb,amsthm}
\usepackage{mathtools}
\usepackage{physics}
\usepackage{geometry}
\geometry{margin=2.5cm}
\usepackage{hyperref}
\usepackage{enumitem}
\usepackage{booktabs}
\usepackage{graphicx}
\usepackage{xcolor}
\usepackage{tcolorbox}

\tcbuselibrary{breakable}
\newtcolorbox{keybox}[1][]{
  colback=blue!5!white,
  colframe=blue!50!black,
  fonttitle=\bfseries,
  title={#1},
  breakable
}
\newtcolorbox{warnbox}[1][]{
  colback=red!5!white,
  colframe=red!50!black,
  fonttitle=\bfseries,
  title={#1},
  breakable
}

\newcommand{\Mpl}{M_{\mathrm{pl}}}
\newcommand{\avg}[1]{\langle #1 \rangle}

\title{Oscillon Formation via Tachyonic Resonance\\[6pt]
\large A Detailed Analysis for the $\alpha$-Attractor T-Model\\
with Application to Numerical Simulations}
\author{Working Notes}
\date{\today}

% ══════════════════════════════════════════════════════════════════════════════
\begin{document}
\maketitle
\tableofcontents
\newpage

% ══════════════════════════════════════════════════════════════════════════════
\section{The Scalar Field Potential}
\label{sec:potential}

We work with the $\alpha$-attractor T-model potential
\cite{Kallosh:2013hoa,Amin:2019ums}:
\begin{equation}
\label{eq:potential}
  V(\varphi) = \frac{1}{2}\,m^{2}\mu^{2}
  \Bigl(1 - e^{\,\varphi/\mu}\Bigr)^{2}\,,
\end{equation}
where $m$ is the scalar field mass (setting the oscillation frequency) and
$\mu$ is the self-interaction scale that controls the nonlinearity of the
potential.
In our code units we set $8\pi G = 1$ and $m = 1$.
The code parameter \texttt{selfinteraction} $= \mu = 0.06$.

\subsection{Key properties}

\begin{enumerate}
  \item \textbf{Minimum:} $V(0) = 0$, $V'(0) = 0$.
  \item \textbf{Mass at the minimum:}
    \begin{equation}
      V''(0) = m^{2}\bigl(2 - 1\bigr) = m^{2} = 1\,,
    \end{equation}
    confirming the oscillation frequency $\omega \approx m = 1$.
  \item \textbf{Plateau:} For $\varphi \to -\infty$,
    $V \to \tfrac{1}{2}\,m^{2}\mu^{2}$.
    With our parameters: $V_{\mathrm{plat}} = \tfrac{1}{2}(0.06)^{2}
    = 1.8\times 10^{-3}$.
  \item \textbf{Steep wall:} For $\varphi > 0$,
    $V$ grows exponentially as $\sim e^{2\varphi/\mu}$.
\end{enumerate}

The second derivative of the potential is:
\begin{equation}
\label{eq:Vpp}
\boxed{
  V''(\varphi) = m^{2}\,e^{\,\varphi/\mu}
  \Bigl(2\,e^{\,\varphi/\mu} - 1\Bigr)\,.
}
\end{equation}
This is positive (stable oscillation) near $\varphi = 0$, but becomes
\emph{negative} (tachyonic) for sufficiently negative $\varphi$.

% ══════════════════════════════════════════════════════════════════════════════
\section{The Tachyonic Region}
\label{sec:tachyonic}

\subsection{Finding the threshold}

The tachyonic region is the set of field values where $V''(\varphi) < 0$.
From Eq.~\eqref{eq:Vpp}:
\begin{equation}
  V''(\varphi) < 0
  \quad\Longleftrightarrow\quad
  2\,e^{\,\varphi/\mu} - 1 < 0
  \quad\Longleftrightarrow\quad
  e^{\,\varphi/\mu} < \tfrac{1}{2}
  \quad\Longleftrightarrow\quad
  \varphi < -\mu\ln 2\,.
\end{equation}

\begin{keybox}[Tachyonic threshold]
\begin{equation}
\label{eq:tach_threshold}
  \varphi_{\mathrm{tach}} = -\mu \ln 2\,.
\end{equation}
With $\mu = 0.06$:
$\quad\varphi_{\mathrm{tach}} = -0.06 \times 0.693 = -0.0416$.

The field must oscillate past this value (to more negative $\varphi$) for
tachyonic instability to occur.
\end{keybox}

\subsection{Maximum tachyonic strength}

We want the extremum of $V''$ in the tachyonic region.
Differentiating $V''$:
\begin{equation}
  \frac{dV''}{d\varphi}
  = \frac{m^{2}}{\mu}\,e^{\,\varphi/\mu}
    \Bigl(4\,e^{\,\varphi/\mu} - 1\Bigr)\,.
\end{equation}
Setting this to zero (excluding the trivial $e^{\varphi/\mu}=0$ limit):
\begin{equation}
  4\,e^{\,\varphi/\mu} = 1
  \quad\Longrightarrow\quad
  \varphi_{\min}^{V''} = -\mu\ln 4 = -2\mu\ln 2\,.
\end{equation}
Substituting back:
\begin{equation}
\label{eq:Vpp_min}
\boxed{
  V''_{\min}
  = m^{2}\cdot\tfrac{1}{4}\bigl(\tfrac{1}{2} - 1\bigr)
  = -\,\frac{m^{2}}{8}\,.
}
\end{equation}
For $m = 1$: $V''_{\min} = -0.125$.

\subsection{Tachyonic band in Fourier space}

A homogeneous perturbation
$\delta\varphi \propto e^{i\mathbf{k}\cdot\mathbf{x}}$ obeys
(in the flat-space, instantaneous approximation):
\begin{equation}
  \ddot{\delta\varphi}_{k}
  + \bigl(k^{2} + V''(\bar{\varphi}(t))\bigr)\,\delta\varphi_{k} = 0\,.
\end{equation}
When $V''(\bar\varphi) < 0$, modes with
\begin{equation}
\label{eq:tach_band}
\boxed{
  k^{2} < \bigl|V''(\bar\varphi)\bigr|
}
\end{equation}
experience \emph{exponential growth} (the effective mass-squared is negative).
The most dangerous instant is when $|V''|$ is maximised, giving a
maximum band width:
\begin{equation}
  k_{\max} = \sqrt{|V''_{\min}|} = \frac{m}{2\sqrt{2}}
  \approx 0.354\,m\,.
\end{equation}

\begin{keybox}[Tachyonic band]
For the $\alpha$-attractor T-model with $\mu = 0.06$ and $m = 1$, modes
with comoving wavenumber
\begin{equation}
  k \lesssim 0.35
\end{equation}
(i.e.\ wavelength $\lambda \gtrsim 18$) can be tachyonically amplified.
The most unstable mode is $k = 0$.
\end{keybox}

% ══════════════════════════════════════════════════════════════════════════════
\section{The Background Oscillation}
\label{sec:background}

\subsection{Energy budget and turning points}

In an FLRW universe with Hubble rate $H$, the homogeneous scalar field
obeys:
\begin{equation}
  \ddot{\bar\varphi} + 3H\dot{\bar\varphi} + V'(\bar\varphi) = 0\,.
\end{equation}
For $H \ll m$, the field oscillates quasi-periodically around $\varphi = 0$
with slowly decaying amplitude.
At any instant the ``energy'' is:
\begin{equation}
  E = \tfrac{1}{2}\dot{\bar\varphi}^{2} + V(\bar\varphi)\,.
\end{equation}
The turning points (where $\dot{\bar\varphi} = 0$) satisfy
$V(\varphi_{\mathrm{turn}}) = E$.

\subsection{Your simulation parameters}

With the initial conditions in the notebook:
\begin{center}
\begin{tabular}{lll}
  \toprule
  Parameter & Symbol & Value \\
  \midrule
  Initial field & $\bar\varphi_0$ & $-0.0864$ \\
  Initial velocity & $\dot{\bar\varphi}_0$ & $+0.0324$ \\
  Self-interaction & $\mu$ & $0.06$ \\
  Scalar mass & $m$ & $1$ \\
  \bottomrule
\end{tabular}
\end{center}

\paragraph{Computing the energy:}
\begin{align}
  E &= \tfrac{1}{2}(0.0324)^{2}
     + \tfrac{1}{2}(0.06)^{2}
       \bigl(1 - e^{-0.0864/0.06}\bigr)^{2}
  \notag\\[4pt]
    &= 5.25\times 10^{-4}
     + 1.8\times 10^{-3}\times(1 - e^{-1.44})^{2}
  \notag\\[4pt]
    &= 5.25\times 10^{-4}
     + 1.8\times 10^{-3}\times(0.763)^{2}
  \notag\\[4pt]
    &= 5.25\times 10^{-4} + 1.05\times 10^{-3}
     \approx 1.57\times 10^{-3}\,.
\end{align}

\paragraph{Negative turning point:}
Setting $V(\varphi_{-}) = E$:
\begin{align}
  \tfrac{1}{2}(0.06)^{2}\bigl(1 - e^{\varphi_{-}/0.06}\bigr)^{2}
  &= 1.57\times 10^{-3}
  \notag\\[3pt]
  \bigl|1 - e^{\varphi_{-}/0.06}\bigr| &= 0.935
  \notag\\[3pt]
  e^{\varphi_{-}/0.06} &= 0.065
  \notag\\[3pt]
  \varphi_{-} &= 0.06\ln(0.065) = -0.164\,.
\end{align}

\paragraph{Positive turning point:}
\begin{equation}
  e^{\varphi_{+}/0.06} = 1.935
  \quad\Longrightarrow\quad
  \varphi_{+} = 0.06\ln(1.935) = +0.040\,.
\end{equation}

\begin{keybox}[Oscillation range]
The background field oscillates between
$\varphi_{-} \approx -0.164$ and $\varphi_{+} \approx +0.040$.

Since $\varphi_{-} = -0.164 < \varphi_{\mathrm{tach}} = -0.042$,
the field \textbf{does} enter the tachyonic region each oscillation.

The ratio $|\varphi_{-}|/\mu = 2.73$ tells us the field explores
$\sim 2.7$ self-interaction lengths on the negative side.
\end{keybox}

\subsection{Evaluating $V''$ at the turning point}

At $\varphi_{-} = -0.164$:
\begin{align}
  V''(-0.164) &= e^{-0.164/0.06}\bigl(2\,e^{-0.164/0.06} - 1\bigr)
  \notag\\[3pt]
  &= 0.065\times(0.130 - 1)
  = 0.065\times(-0.870)
  = -0.057\,.
\end{align}
This is weaker than the theoretical maximum $|V''_{\min}| = 0.125$
because the field doesn't reach $\varphi = -2\mu\ln 2 = -0.083$
with enough ``room'' to spare.
Actually it does pass through $-0.083$; the maximum $|V''|$
experienced is the full $0.125$.

% ══════════════════════════════════════════════════════════════════════════════
\section{Hubble Damping and the Resonance Window}
\label{sec:hubble}

This is the critical section: even though the tachyonic instability exists,
it only operates while the background oscillation amplitude is above the
tachyonic threshold.
Hubble friction steadily reduces the amplitude.

\subsection{Hubble rate}

From the Friedmann equation (with $8\pi G = 1$):
\begin{equation}
  3H^{2} = \rho
  \quad\Longrightarrow\quad
  H = \sqrt{\frac{\rho}{3}}
  \approx \sqrt{\frac{1.57\times 10^{-3}}{3}}
  \approx 0.023\,.
\end{equation}

The ratio $H/m = 0.023 \ll 1$ confirms we are in the
``many oscillations per Hubble time'' regime, so the WKB /
adiabatic treatment is valid.

\subsection{Amplitude decay}

For a massive scalar field oscillating in an expanding universe,
the oscillation-averaged amplitude decays as
\cite{Turner:1983he,Amin:2010dc}:
\begin{equation}
\label{eq:amp_decay}
  |\bar\varphi_{\mathrm{amp}}(t)|
  \propto a(t)^{-3/2}\,,
\end{equation}
where $a(t)$ is the scale factor.
This holds when $H \ll m$ (rapid oscillation limit)
and the equation of state averages to $w = 0$ (matter-like).

\subsection{Time until the resonance shuts off}

The tachyonic instability requires
$|\bar\varphi_{\mathrm{amp}}| > |\varphi_{\mathrm{tach}}| = 0.042$.
Starting from $|\bar\varphi_{\mathrm{amp}}(0)| = 0.164$:
\begin{align}
  0.164\times a^{-3/2} &= 0.042
  \notag\\[3pt]
  a^{3/2} &= \frac{0.164}{0.042} = 3.90
  \notag\\[3pt]
  a &= 3.90^{2/3} = 2.48\,.
\end{align}
The number of $e$-folds:
\begin{equation}
  N = \ln a = \ln 2.48 = 0.91\,.
\end{equation}
The coordinate time (assuming $a \approx e^{Ht}$ during
matter domination\footnote{
  More precisely, $a \propto t^{2/3}$ during matter domination,
  but for $N \lesssim 1$, $a \approx e^{Ht}$ is a reasonable
  approximation for estimating the time scale.
}):
\begin{equation}
\label{eq:t_shutoff}
\boxed{
  \Delta t_{\mathrm{res}} \approx \frac{N}{H}
  = \frac{0.91}{0.023} \approx 40\,.
}
\end{equation}

\begin{warnbox}[Resonance window]
The tachyonic resonance is only active for the first
$\Delta t \approx 40$ code time units, corresponding to roughly
\begin{equation}
  n_{\mathrm{osc}} = \frac{\Delta t}{T}
  = \frac{40}{2\pi} \approx 6 \text{ oscillations}\,.
\end{equation}
After this, the background amplitude drops below the tachyonic threshold
and the instability switches off permanently.
Any perturbation growth must happen within this window.
\end{warnbox}

% ══════════════════════════════════════════════════════════════════════════════
\section{Growth Rate of Perturbations}
\label{sec:growth}

\subsection{Instantaneous growth rate}

During the tachyonic phase (when $V'' < 0$ and $k^{2} < |V''|$),
a Fourier mode $\delta\varphi_{k}$ grows as
\cite{Amin:2011hj,Lozanov:2017hjm}:
\begin{equation}
  \delta\varphi_{k} \propto
  \exp\!\left(\int_{\mathrm{tach}}
  \sqrt{|V''(\bar\varphi(t'))| - k^{2}}\;dt'\right).
\end{equation}

\subsection{Floquet exponent estimate}

For each oscillation period, the field spends a fraction $f$ of the
time in the tachyonic region.
Let $\langle\sqrt{|V''|}\rangle_{\mathrm{tach}}$ be the time-averaged
value of $\sqrt{|V''|}$ during the tachyonic phase.
Then the Floquet exponent per oscillation is approximately:
\begin{equation}
  \mu_{F} \approx f \cdot T \cdot
  \langle\sqrt{|V''|}\rangle_{\mathrm{tach}}\,,
\end{equation}
where $T = 2\pi/m \approx 6.28$.

\paragraph{Estimating $f$:}
The tachyonic region is $\varphi < -0.042$.
The full swing on the negative side goes from $0$ to $-0.164$.
The fraction of the negative half-swing that is tachyonic is
$(0.164 - 0.042)/0.164 \approx 0.74$.
The field spends roughly half the period on the negative side,
so $f \approx 0.5 \times 0.74 \approx 0.37$.
However, the field moves faster near $\varphi = 0$ and slower near the
turning points, so the \emph{time} fraction is larger than the
spatial fraction.
A reasonable estimate is $f \approx 0.30$--$0.40$.

\paragraph{Estimating $\langle\sqrt{|V''|}\rangle$:}
$|V''|$ varies from $0$ at the threshold to $\sim 0.125$ at $\varphi_{\min}^{V''}$.
A geometric mean gives
$\langle\sqrt{|V''|}\rangle \approx \sqrt{0.125/2} \approx 0.25$.

\paragraph{Result:}
\begin{equation}
\label{eq:floquet}
  \mu_{F} \approx 0.35 \times 6.28 \times 0.25 \approx 0.55\,.
\end{equation}
Growth per oscillation:
\begin{equation}
  G_{\mathrm{osc}} = e^{\mu_{F}} \approx e^{0.55} \approx 1.73\,.
\end{equation}

\begin{keybox}[Total amplification]
Over $n_{\mathrm{osc}} \approx 6$ oscillations:
\begin{equation}
\label{eq:total_growth}
  G_{\mathrm{total}} = G_{\mathrm{osc}}^{\,n_{\mathrm{osc}}}
  = 1.73^{6} \approx 27\,.
\end{equation}
The perturbation amplitude grows by a factor of $\sim 27$ before
the resonance shuts off.
\end{keybox}

% ══════════════════════════════════════════════════════════════════════════════
\section{Will an Oscillon Form? A Criterion}
\label{sec:criterion}

\subsection{The nonlinearity condition}

For an oscillon to form, the perturbation must reach
\emph{nonlinear} amplitudes, meaning $\delta\varphi \sim \mu$
(the self-interaction scale) or equivalently $\delta\rho/\bar\rho \sim 1$
\cite{Amin:2011hj,Amin:2019ums}.

\begin{keybox}[Oscillon formation criterion]
An oscillon forms if:
\begin{equation}
\label{eq:criterion}
\boxed{
  A_{\mathrm{pert}} \times G_{\mathrm{total}} \gtrsim \mu\,,
}
\end{equation}
where $A_{\mathrm{pert}}$ is the initial perturbation amplitude,
$G_{\mathrm{total}}$ is the total growth factor during the resonance
window, and $\mu$ is the self-interaction scale.
\end{keybox}

\subsection{Applying to your simulation}

With your current parameters:
\begin{equation}
  A_{\mathrm{pert}} \times G_{\mathrm{total}}
  = 10^{-3} \times 27
  = 0.027\,.
\end{equation}
Comparing to $\mu = 0.06$:
\begin{equation}
  0.027 < 0.06 = \mu\,.
\end{equation}
The perturbation does \emph{not} reach the nonlinearity scale.
It falls short by about a factor of 2.

\subsection{What would work?}

Rearranging Eq.~\eqref{eq:criterion}:
\begin{equation}
  A_{\mathrm{pert}} \gtrsim \frac{\mu}{G_{\mathrm{total}}}
  = \frac{0.06}{27} \approx 2.2\times 10^{-3}\,.
\end{equation}

So an initial perturbation amplitude of
$A \gtrsim 3\times 10^{-3}$ should be sufficient.
However, this estimate is rough; accounting for Hubble dilution of the
perturbation itself and imperfect overlap with the resonance band,
a safer choice is:
\begin{equation}
  A_{\mathrm{pert}} \sim 10^{-2}\text{--}10^{-1}\,.
\end{equation}

% ══════════════════════════════════════════════════════════════════════════════
\section{The Role of the Perturbation Width}
\label{sec:width}

\subsection{Fourier content of a Gaussian bump}

A Gaussian perturbation in real space,
\begin{equation}
  \delta\varphi(r) = A\,\exp\!\Bigl(-\frac{r^{2}}{2R^{2}}\Bigr),
\end{equation}
has a Fourier transform (in 3D) that peaks at $k = 0$ with power
falling off as:
\begin{equation}
  |\widetilde{\delta\varphi}(k)|^{2} \propto
  \exp\!\bigl(-k^{2}R^{2}\bigr)\,.
\end{equation}

The fraction of power at wavenumber $k$ relative to the peak
is $\exp(-k^{2}R^{2})$.

\subsection{Overlap with the tachyonic band}

The tachyonic band extends to $k_{\max} \approx 0.35$.
For different choices of $R$:

\begin{center}
\begin{tabular}{cccc}
  \toprule
  $R$ & $k_{\max}R$ & Power at $k_{\max}$: $e^{-k_{\max}^2 R^2}$
      & Assessment \\
  \midrule
  $1.0$ & $0.35$ & $0.88$ & Good \\
  $1.5$ & $0.53$ & $0.76$ & Reasonable \\
  $3.0$ & $1.05$ & $0.33$ & Still OK \\
  $5.0$ & $1.75$ & $0.047$ & Poor overlap \\
  \bottomrule
\end{tabular}
\end{center}

\begin{keybox}[Optimal perturbation width]
For the tachyonic band with $k_{\max} \approx 0.35$:
\begin{itemize}
  \item $R \lesssim 3$ keeps substantial power in the resonance band.
  \item The \emph{optimal} choice is $R \sim 1/k_{\max} \approx 3$,
    which balances having power at the most unstable modes ($k \approx 0$)
    with resolving the perturbation on the grid.
  \item Your choice $R = 1.5$ is fine; most of the Fourier power
    is within the tachyonic band.
\end{itemize}

However, note the distinction between the tachyonic band and
\textbf{parametric resonance} bands (Sec.~\ref{sec:parametric}),
which operate at higher $k$.
\end{keybox}

\subsection{The supervisor's comment: ``order 1 wavelength''}

Your supervisor said perturbations of ``order $1/m$ wavelength''
are subject to resonance.
This refers to \emph{parametric} (Mathieu-type) resonance, not
tachyonic resonance.
For parametric resonance, the instability bands are centred at
$k \approx n\omega/2$ ($n = 1, 2, \ldots$), giving the first band at
$k \sim m/2 = 0.5$ (wavelength $\lambda = 2\pi/k \approx 12.6$).

For a Gaussian with $R = 1.5$, the power at $k = 0.5$ is
$e^{-0.25 \times 2.25} = e^{-0.56} \approx 0.57$ --- decent overlap.

The key point: \textbf{both} tachyonic and parametric resonance
operate, but they dominate at different $k$ ranges:
\begin{itemize}
  \item \textbf{Tachyonic:} $k \lesssim 0.35$
    (broader in real space, $\lambda \gtrsim 18$).
  \item \textbf{Parametric:} $k \sim 0.5$
    (narrower in real space, $\lambda \sim 12$).
\end{itemize}

% ══════════════════════════════════════════════════════════════════════════════
\section{Parametric Resonance (Floquet Theory)}
\label{sec:parametric}

When $V''(\bar\varphi(t))$ oscillates periodically, even if it never goes
negative, modes can still grow via \emph{parametric resonance}
(the Mathieu instability).
This is relevant because even after the background amplitude drops below
$\varphi_{\mathrm{tach}}$, parametric resonance can continue to operate.

The perturbation equation can be written in Mathieu form:
\begin{equation}
  \frac{d^{2}\delta\varphi_{k}}{dz^{2}}
  + \bigl(A_{k} - 2q\cos 2z\bigr)\,\delta\varphi_{k} = 0\,,
\end{equation}
where $z = mt/2$ and:
\begin{equation}
  A_{k} = \frac{k^{2} + m^{2}}{(m/2)^{2}} = 4\frac{k^{2}+m^{2}}{m^{2}}\,,
  \qquad
  q \propto \frac{\varphi_{\mathrm{amp}}}{\mu}\,.
\end{equation}
Instability bands occur near $A_{k} = n^{2}$ ($n = 1, 2, \ldots$):
\begin{itemize}
  \item $n = 1$: $k \approx 0$ (overlaps with tachyonic band).
  \item $n = 2$: $k \approx m$ (the ``order 1 wavelength'' modes).
\end{itemize}

The parametric growth rate is generally weaker than the tachyonic
rate but persists longer (it doesn't require the field to reach the
tachyonic region).
See \cite{Kofman:1997yn,Amin:2011hj} for details.

% ══════════════════════════════════════════════════════════════════════════════
\section{Putting It All Together: When Can Oscillons Form?}
\label{sec:summary}

\subsection{The three requirements}

\begin{enumerate}[label=\textbf{\arabic*.}]
  \item \textbf{The field must access the tachyonic region:}
    $|\bar\varphi_{\mathrm{amp}}| > |\varphi_{\mathrm{tach}}| = \mu\ln 2$.
    This requires sufficient initial energy:
    \begin{equation}
      E > V(\varphi_{\mathrm{tach}})
      = \tfrac{1}{2}m^{2}\mu^{2}\bigl(1 - \tfrac{1}{2}\bigr)^{2}
      = \tfrac{m^{2}\mu^{2}}{8}\,.
    \end{equation}

  \item \textbf{Enough oscillations in the resonance window:}
    Hubble damping reduces the amplitude as $a^{-3/2}$.
    The number of oscillations before shutdown is:
    \begin{equation}
      n_{\mathrm{osc}} \approx
      \frac{m}{2\pi H}\ln\!\left(
      \frac{|\bar\varphi_{\mathrm{amp},0}|}{|\varphi_{\mathrm{tach}}|}
      \right)^{2/3}.
    \end{equation}
    More oscillations $\Rightarrow$ more growth.
    This favours:
    \begin{itemize}
      \item Large initial amplitude $|\bar\varphi_{\mathrm{amp},0}|/\mu$.
      \item Small $H/m$ ratio.
    \end{itemize}

  \item \textbf{Sufficient initial perturbation amplitude:}
    \begin{equation}
      A_{\mathrm{pert}} \gtrsim
      \frac{\mu}{G_{\mathrm{osc}}^{\,n_{\mathrm{osc}}}}\,.
    \end{equation}
\end{enumerate}

\subsection{Parameter study for your simulation}

\begin{center}
\renewcommand{\arraystretch}{1.3}
\begin{tabular}{lccl}
  \toprule
  \textbf{Knob} & \textbf{Current} & \textbf{Suggested}
  & \textbf{Effect} \\
  \midrule
  $A_{\mathrm{pert}}$ & $10^{-3}$ & $10^{-2}$--$10^{-1}$
  & Gives head start; $A\times G > \mu$ \\
  $R$ & $1.5$ & $1.5$--$3.0$
  & More power in tachyonic band \\
  $u_{\mathrm{val}}$ & $-0.0864$ & $-0.15$ to $-0.3$
  & Larger oscillation, more tachyonic cycles \\
  $v_{\mathrm{val}}$ & $+0.0324$ & scale with energy
  & Consistent with larger amplitude \\
  \bottomrule
\end{tabular}
\end{center}

\begin{warnbox}[Most impactful change]
The \textbf{single most impactful change} is increasing
$A_{\mathrm{pert}}$ from $10^{-3}$ to $\sim 10^{-2}$.
With the current background ($G_{\mathrm{total}} \approx 27$):
\begin{equation}
  10^{-2} \times 27 = 0.27 \gg \mu = 0.06 \quad\checkmark
\end{equation}
This should allow oscillon formation.

The second most impactful change is using a larger initial
background amplitude $|u_{\mathrm{val}}|$, which increases
$n_{\mathrm{osc}}$ and therefore $G_{\mathrm{total}}$
exponentially.
For example, doubling $|u_{\mathrm{val}}|$ to $0.17$ gives roughly
$n_{\mathrm{osc}} \approx 10$, so $G_{\mathrm{total}} \approx 1.73^{10}
\approx 240$, making even $A = 10^{-4}$ sufficient.
\end{warnbox}

% ══════════════════════════════════════════════════════════════════════════════
\section{Diagnostic Checklist for Oscillon Formation}
\label{sec:diagnostics}

Once you run a simulation, here is how to determine whether an oscillon
has formed:

\begin{enumerate}[label=\textbf{D\arabic*.}]
  \item \textbf{Density contrast $\delta_{c}(t)$:}
    $\delta_{c} = \rho_{c}/\bar\rho - 1$ should grow during the
    resonance window ($t \lesssim 40$), then \emph{saturate and
    oscillate} rather than decay.
    A successful oscillon has $\delta_{c} \gg 1$ and growing
    as $\propto a^{3}$ (since $\bar\rho \propto a^{-3}$ while
    $\rho_{c}$ stays roughly constant).

  \item \textbf{Central density $\rho_{c}(t)$:}
    Should stabilise to an approximately constant value
    (modulated by field oscillations at frequency $\sim m$)
    rather than decaying as $a^{-3}$.

  \item \textbf{Density profile $\rho(r)$:}
    A clear localised peak at $r = 0$ with radius
    $R_{\mathrm{osc}} \sim \pi/m \approx 3$.

  \item \textbf{Compactness $C = GM_{\mathrm{osc}}/R_{\mathrm{osc}}$:}
    Should be non-zero and stable.
    Typical values $C \sim 10^{-3}$--$10^{-1}$.

  \item \textbf{Scalar field at centre $\varphi(r=0, t)$:}
    Should show sustained large-amplitude oscillations
    even as the background decays.
    Plotting $\varphi(0,t) - \bar\varphi(t)$ should reveal
    a growing or sustained perturbation.
\end{enumerate}

% ══════════════════════════════════════════════════════════════════════════════
\section{Quick Reference Formulas}
\label{sec:formulas}

For the $\alpha$-attractor T-model
$V = \frac{1}{2}m^{2}\mu^{2}(1 - e^{\varphi/\mu})^{2}$:

\begin{center}
\renewcommand{\arraystretch}{1.6}
\begin{tabular}{ll}
  \toprule
  \textbf{Quantity} & \textbf{Formula} \\
  \midrule
  Tachyonic threshold
  & $\varphi_{\mathrm{tach}} = -\mu\ln 2$ \\
  Max $|V''|$
  & $m^{2}/8$, at $\varphi = -2\mu\ln 2$ \\
  Tachyonic band
  & $k < m/(2\sqrt{2})$ \\
  Oscillation period
  & $T = 2\pi/m$ \\
  Amplitude decay
  & $|\varphi_{\mathrm{amp}}| \propto a^{-3/2}$ \\
  Hubble rate
  & $H = \sqrt{\rho/3}$ \quad(with $8\pi G = 1$) \\
  Resonance window
  & $\Delta t \approx \frac{2}{3H}
    \ln\bigl(|\varphi_{\mathrm{amp},0}|/|\varphi_{\mathrm{tach}}|\bigr)$ \\
  Resonance oscillations
  & $n_{\mathrm{osc}} = m\Delta t/(2\pi)$ \\
  Growth per oscillation
  & $G_{\mathrm{osc}} \approx e^{f T \langle\sqrt{|V''|}\rangle}
    \approx 1.7$ \\
  Oscillon criterion
  & $A_{\mathrm{pert}} \times G_{\mathrm{osc}}^{n_{\mathrm{osc}}}
    \gtrsim \mu$ \\
  \bottomrule
\end{tabular}
\end{center}

% ══════════════════════════════════════════════════════════════════════════════
\section{Summary of the Problem and Fix}
\label{sec:fix}

\textbf{Why oscillons did not form in your simulation:}

\begin{enumerate}
  \item The tachyonic resonance window is limited to $\sim 6$ oscillations
    ($\Delta t \approx 40$) before Hubble damping reduces the background
    amplitude below the tachyonic threshold.
  \item The total growth factor is $G \approx 27$.
  \item Starting from $A = 10^{-3}$, the perturbation only reaches
    $\sim 0.027$, which is below the nonlinearity scale $\mu = 0.06$.
  \item After $t = 40$, the resonance stops and the perturbation
    (now sub-threshold for nonlinear collapse) simply dilutes with
    the expansion.
  \item Your simulation ran to $t = 400$, but the resonance was already
    dead by $t \approx 40$.
    The remaining 360 time units showed no further growth.
\end{enumerate}

\textbf{Fix:}
Increase \texttt{perturbation\_amplitude} to $10^{-2}$ or larger.
This ensures $A \times G \approx 0.27 > \mu = 0.06$, allowing the
perturbation to reach nonlinear amplitudes within the resonance window.

% ══════════════════════════════════════════════════════════════════════════════

\begin{thebibliography}{99}

\bibitem{Kallosh:2013hoa}
  R.~Kallosh, A.~Linde, and D.~Roest,
  ``Superconformal Inflationary $\alpha$-Attractors,''
  JHEP \textbf{1311}, 198 (2013),
  \href{https://arxiv.org/abs/1311.0472}{arXiv:1311.0472}.

\bibitem{Amin:2011hj}
  M.~A.~Amin, R.~Easther, H.~Finkel, R.~Flauger, and M.~P.~Hertzberg,
  ``Oscillons After Inflation,''
  Phys.\ Rev.\ Lett.\ \textbf{108}, 241302 (2012),
  \href{https://arxiv.org/abs/1106.3335}{arXiv:1106.3335}.

\bibitem{Amin:2019ums}
  M.~A.~Amin,
  ``Inflaton fragmentation: Emergence of pseudo-stable inflaton lumps
  (oscillons) after inflation,''
  \href{https://arxiv.org/abs/1006.3075}{arXiv:1006.3075}.

\bibitem{Lozanov:2017hjm}
  K.~D.~Lozanov and M.~A.~Amin,
  ``Self-resonance after inflation: oscillons, transients, and
  radiation domination,''
  Phys.\ Rev.\ D \textbf{97}, 023533 (2018),
  \href{https://arxiv.org/abs/1710.06851}{arXiv:1710.06851}.

\bibitem{Kofman:1997yn}
  L.~Kofman, A.~Linde, and A.~A.~Starobinsky,
  ``Towards the theory of reheating after inflation,''
  Phys.\ Rev.\ D \textbf{56}, 3258 (1997),
  \href{https://arxiv.org/abs/hep-ph/9704452}{arXiv:hep-ph/9704452}.

\bibitem{Turner:1983he}
  M.~S.~Turner,
  ``Coherent scalar field oscillations in an expanding universe,''
  Phys.\ Rev.\ D \textbf{28}, 1243 (1983).

\bibitem{Amin:2010dc}
  M.~A.~Amin and D.~Shiber,
  ``Formation, gravitational clustering, and interactions of
  nonrelativistic solitons in an expanding universe,''
  \href{https://arxiv.org/abs/1210.4979}{arXiv:1210.4979}.

\bibitem{Clough:2023oscillon}
  K.~Clough, E.~A.~Lim, et al.,
  ``Oscillon formation in full general relativity,''
  (2023),
  \href{https://arxiv.org/abs/2304.01673}{arXiv:2304.01673}.

\end{thebibliography}

\end{document}
